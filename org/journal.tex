% Created 2018-11-23 Fri 09:30
% Intended LaTeX compiler: xelatex
\documentclass[a4paper]{article}
\usepackage{graphicx}
\usepackage{grffile}
\usepackage{longtable}
\usepackage{wrapfig}
\usepackage{rotating}
\usepackage[normalem]{ulem}
\usepackage{amsmath}
\usepackage{textcomp}
\usepackage{amssymb}
\usepackage{capt-of}
\usepackage{hyperref}
\usepackage[margin=2.5cm]{geometry}
\usepackage{fontspec}
\setmainfont{Calibri}
\author{Sébastien Guyader}
\date{\today}
\title{Journal}
\hypersetup{
 pdfauthor={Sébastien Guyader},
 pdftitle={Journal},
 pdfkeywords={},
 pdfsubject={},
 pdfcreator={Emacs 26.1 (Org mode 9.1.14)}, 
 pdflang={Frenchb}}
\begin{document}

\maketitle
\tableofcontents

informations were gathered and first demonstrated in my [\url{https://github.com/alegrand/RR\_webinars/blob/master/1\_replicable\_article\_laboratory\_notebook/index.org}][First
webinar on reproducible research: litterate programming].
\begin{itemize}
\item Emacs shortcuts
\label{sec:org06aa019}
Here are a few convenient emacs shortcuts for those that have never
used emacs. In all of the emacs shortcuts, \texttt{C=Ctrl}, \texttt{M=Alt/Esc} and
\texttt{S=Shift}.  Note that you may want to use two hours to follow the emacs
tutorial (\texttt{C-h t}). In the configuration file CUA keys have been
activated and allow you to use classical copy/paste (\texttt{C-c/C-v})
shortcuts. This can be changed from the Options menu.
\begin{itemize}
\item \texttt{C-x C-c} exit
\item \texttt{C-x C-s} save buffer
\item \texttt{C-g} panic mode ;) type this whenever you want to exit an awful
series of shortcuts
\item \texttt{C-Space} start selection marker although selection with shift and
arrows should work as well
\item \texttt{C-l} reposition the screen
\item \texttt{C-\_} (or \texttt{C-z} if CUA keys have been activated)
\item \texttt{C-s} search
\item \texttt{M-\%} replace
\item \texttt{C-x C-h} get the list of emacs shortcuts
\item \texttt{C-c C-h} get the list of emacs shortcuts considering the mode you are
currently using (e.g., C, Lisp, org, \ldots{})
\item With the "\emph{reproducible research}" emacs configuration, \texttt{C-x g} allows
you to invoke \href{https://magit.vc/}{Magit} (provided you installed it beforehand!) which
is a nice git interface for Emacs.
\end{itemize}
There are a bunch of cheatsheets also available out there (e.g.,
\href{http://www.shortcutworld.com/en/linux/Emacs\_23.2.1.html}{this one for emacs} and \href{http://orgmode.org/orgcard.txt}{this one for org-mode} or this \href{http://sachachua.com/blog/wp-content/uploads/2013/05/How-to-Learn-Emacs-v2-Large.png}{graphical one}).
\item Org-mode
\label{sec:orgf32b592}
Many emacs shortcuts start by \texttt{C-x}. Org-mode's shortcuts generaly
start with \texttt{C-c}.
\begin{itemize}
\item \texttt{Tab} fold/unfold
\item \texttt{C-c c} capture (finish capturing with \texttt{C-c C-c}, this is explained on
the top of the buffer that just opened)
\item \texttt{C-c C-c} do something useful here (tag, execute, \ldots{})
\item \texttt{C-c C-o} open link
\item \texttt{C-c C-t} switch todo
\item \texttt{C-c C-e} export
\item \texttt{M-Enter} new item/section
\item \texttt{C-c a} agenda (try the \texttt{L} option)
\item \texttt{C-c C-a} attach files
\item \texttt{C-c C-d} set a deadl1ine (use \texttt{S-arrows} to navigate in the dates)
\item \texttt{A-arrows} move subtree (add shift for the whole subtree)
\item table des matières : ajouter  \texttt{\#+TOC: headlines 2} à l'endroit
souhaité (avec ici 2 niveaux)
\end{itemize}
\item Org-mode Babel (for literate programming)
\label{sec:orgcec9a19}
\begin{itemize}
\item \texttt{<s + tab} template for source bloc. You can easily adapt it to get
this:
\begin{verbatim}
#+begin_src shell
ls
#+end_src
\end{verbatim}
Now if you \texttt{C-c C-c}, it will execute the block.
\begin{verbatim}
#+RESULTS:
| #journal.org# |
| journal.html  |
| journal.org   |
| journal.org~  |
\end{verbatim}

\item Source blocks have many options (formatting, arguments, names,
sessions,\ldots{}), which is why I have my own shortcuts \texttt{<b + tab} bash
block (or \texttt{B} for sessions).
\begin{verbatim}
#+begin_src shell :results output :exports both
ls /tmp/*201*.pdf
#+end_src

#+RESULTS:
: /tmp/2015_02_bordeaux_otl_tutorial.pdf
: /tmp/2015-ASPLOS.pdf
: /tmp/2015-Europar-Threadmap.pdf
: /tmp/europar2016-1.pdf
: /tmp/europar2016.pdf
: /tmp/M2-PDES-planning-examens-janvier2016.pdf
\end{verbatim}
\item I have defined many such templates in my configuration. You can
give a try to \texttt{<r}, \texttt{<R}, \texttt{<RR}, \texttt{<g}, \texttt{<p}, \texttt{<P}, \texttt{<m} \ldots{}
\item Some of these templates are not specific to babel: e.g., \texttt{<h}, \texttt{<l},
\texttt{<L}, \texttt{<c}, \texttt{<e}, \ldots{}
\end{itemize}
turday
\end{itemize}

\section*{2018}
\label{sec:org008afda}
\subsection*{2018-11 November}
\label{sec:org27a5c66}
\subsubsection*{2018-11-17 Saturday}
\label{sec:org30c4845}
\begin{itemize}
\item \textbf{Installation et configuration d'Emacs org-mode pour ce journal}
\label{sec:org3a3ad8c}
\end{itemize}
\subsubsection*{2018-11-19 Monday}
\label{sec:orgde93fbf}
\begin{itemize}
\item \textbf{Installation du système complet sur mon mini-pc Windows 10 perso}
\label{sec:org833a60d}

\begin{itemize}
\item notepad++ 7.5.9 (à partir du Windows store pour pouvoir le définir comme éditeur de fichiers .txt par défaut)
\item Git 2.19.1
\item R 3.5.1
\item Python 3.7.1
\item Perl (Strawberry) 5.28.0.1
\item ImageMagick 7.0.8 Q16
\item Ghostscript 9.25 (ajout du répertoire du binaire dans la variable d'environnement PATH)
\item MikTex 2.9.6
\item paquet \TeX{} "pdfcrop"
\item Xournal 0.4.8
\item TexMaker 5.0.3
\item emacs 26.1 (version modifiée pour Windows, obtenue [[\url{https://vigou3.gitlab.io/emacs-modified-windows/}][à partir de cette
\end{itemize}
source]])

\item \textbf{Travail sur un script pour convertir un tableau (aide à F. Causeret pour Cavalbio)}
\label{sec:orga138618}
\end{itemize}

\subsubsection*{2018-11-20 Tuesday}
\label{sec:org542b5f4}
\begin{itemize}
\item \textbf{Script et tableaux finaux pour l'analyse économétrique envoyés à F. Causeret}
\label{sec:orgad8bd90}

\item \textbf{Recyclage SST}
\label{sec:orgde0ac84}

\begin{itemize}
\item Intro :
\begin{itemize}
\item penser à renseigner le "registre santé et sécurité au travail" disponible dans chaque unité, si on a une chose à faire remonter (amélioration, danger potentiel)
\item document "autorisation de sortie" sans infirmière (disponible sur le site intranet prévention, auprès des AP)
\item fiche d'appel d'urgence dans chaque unité
\item site internet www.apo.com
\end{itemize}

\item Plan d'intervention SST

\item numéros d'appel :
\begin{itemize}
\item 15 (SAMU, en priorité si à la maison ou au travail)
\item 18 (pompiers)
\item 17 (gendarmerie, sur voie publique)
\item 112 (numéro européen, y compris en cas d'attaque terroriste)
\item 114 (texto, pour personnes malentendantes)
\item 196 (à terre, si on voit un danger en mer)
\end{itemize}

\item malaise :
\begin{itemize}
\item communiquer, poser des questions
\item AVC : inspecter les signes (déformation visage, douleur tête et bras\ldots{})
\item appel :
\begin{itemize}
\item je me présente, SST, numéro par lequel j'appelle
\item je décris la situation, le nb de victimes, ce que j'ai fait
\item le lieu exact
\item si on envoie une personne à tel endroit
\item attendre que le service nous dise de raccrocher
\end{itemize}
\item si crise d'épilepsie, dégager l'espace et attendre que la crise passe
\end{itemize}

\item PLS :
\begin{itemize}
\item si victime inconsciente qui respire
\item pour dégager les voies respiratoires
\end{itemize}

\item Saignements
\begin{itemize}
\item faire se moucher vigoureusement si saignement de nez, pincer les narines 10 minutes
\item abondant : Arrêter, Allonger, Alerter
\item arrêter par compression ; pansement si on doit s'éloigner ; si pansement imbibé au retour, reprendre compression
\item garrot si besoin de s'absenter pour porter secours à d'autres victimes
\end{itemize}

\item Brûlure
\begin{itemize}
\item thermique ou électrique : arroser par ruissellement
\item inhalat
\end{itemize}
\end{itemize}
\end{itemize}
\subsubsection*{2018-11-21 mercredi}
\label{sec:org4e7bfaa}
\begin{itemize}
\item \textbf{Test de modification du journal sur mini-PC perso}
\label{sec:orgeafc878}

\item \textbf{Améliorations au journal / OrgMode}
\label{sec:orgce505a8}
\end{itemize}
\subsubsection*{2018-11-22 Thursday}
\label{sec:orgea5538b}
\begin{itemize}
\item \textbf{Configuration autour de la recherche reproductible}
\label{sec:orgedd6e1b}
\begin{itemize}
\item lu pas mal de références sur rmarkdown, emacs-OrgMode,
bookdown\ldots{} pour trouver la solution idéale pour la prise de notes
\item installé \texttt{bookdown} mais ça paraît compliqué : pas de gestion au jour
le jour, ou alors il faut créer un chapitre par jour et/ou activité,
ce qui rend ensuite difficile de tout avir sous l'oeil ->
emacs-OrgMode gagne haut la main
\item désinstallé TexLive pour installer TinyTeX sous R/RStudio, mais ça
casse le fonctionnement de \LaTeX{} avec emacs, donc désinstallé
TinyTex et réinstallé TexLive (à partir du site web de TexLive, non
pas à partir du gestionnaire de paquets de la distribution Linux)
\item installé la librarie "citr" pour tester l'intégration de références
biblio directement depuis Zotero dans un fichier rmarkdown
\item à propos de \texttt{bookdown}, j'ai trouvé comment initialiser un nouveau
book :
\begin{itemize}
\item créer un nouveau projet R
\item dans la console, taper : \texttt{bookdown:::bookdown\_skeleton(getwd())}
\item cela va créer un squelette de livre à éditer
\end{itemize}
\end{itemize}

Entered on \textit{[2018-11-22 Thu 20:23]}

\url{file:///home/sguyader/TRAVAIL/github/guyader-lab-inra/org/journal.org}
\end{itemize}
\end{document}