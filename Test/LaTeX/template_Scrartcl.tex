\documentclass[]{article}

\usepackage[pdfstartview=FitH]{hyperref}
%\usepackage[T1]{fontenc}
%\usepackage[sfdefault]{roboto}

\usepackage{fontspec}
\setmainfont{Calibri}

%opening
\title{}
\author{}

\begin{document}

 
\maketitle

\begin{abstract}

Ceci est l'abstract.

\end{abstract}

\hypertarget{module-1}{%
	\section{Module 1}\label{module-1}}

\hypertarget{ce-que-je-retiens}{%
	\subsection{Ce que je retiens}\label{ce-que-je-retiens}}

La partie historique était intéressante. J'ai surtout apprécié la partie
concernant \textbf{l'index-ation}, et les méthodes et outils s'y
rapportant. Je connaissais déjà le langage \texttt{Markdown}, pour
l'avoir utilisé sur d'autres plateformes collaboratives
(\texttt{github}, \texttt{pixls.us}) et sous \texttt{RStudio} avec
\texttt{Rmarkdown} pour les rapports de travaux.

\hypertarget{remarques}{%
	\subsection{Remarques}\label{remarques}}

J'ai démarré une
\href{https://www.fun-mooc.fr/courses/course-v1:inria+41016+session01bis/discussion/forum/c309fd5170d67bde98ab305f18f063d9a55554f8/threads/5bce23551c89dc02ae005933}{discussion}
concernant la première question du Quizz 1 : je pense que la réponse
``outils numériques'' devrait être considérée comme valide, car cet
outil est clairement cité dans
\href{https://d381hmu4snvm3e.cloudfront.net/videos/ZXRQTwenN9LW/SD.mp4}{l'une
	des vidéos} par l'historienne interviewée.
\end{document}
